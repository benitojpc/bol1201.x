\chapter{Aspectos básicos de la inversión bursátil}

\section{Objetivos}

\begin{itemize}
    \item Saber diferenciar entre \ti{operativa a largo plazo} y \ti{operativa a corto plazo}.
    \item Entender que, en el largo plazo, la inversión en bolsa es la mejor opción.
    \item Saber qué efecto puede tener la diversificación en mis inversiones.
    \item Conocer el papel que juegan los dividendos en mis decisiones de inversión.
\end{itemize}

\section{Primeros pasos}

¿Cuál es nuestro objetivo?. Rentabilizar capital, ganando dinero de forma consistente y con un riesgo limitado.

Opciones de activos financieros en los que invertir:
\begin{itemize}
    \item Destinar nuestro capital a un \ti{depósito bancario}. La rentabilidad es muy baja.
    \item Invertir en \ti{renta fija}, es segura y con un riesgo mínimo, su rentabilidad también es bastante baja.
    \item Invertir en \ti{renta variable}, más favorable que las anteriores, nos permite invertir en
    \begin{itemize}
        \item acciones
        \item en CFD's
        \item futuros, \ldots
    \end{itemize}
    \item En \ti{fondos de inversión}, que pueden combinar alguno de los activos que hemos visto.
\end{itemize}

Las dos últimas suelen ser preferibles a las primeras, sobre todo a largo plazo ya que nos darán mayor rentabilidad.

¿Qué tipo de inversor podemos ser?. Básicamente tenemos dos tipos:
\begin{itemize}
    \item \tb{Inversor a largo plazo}, personas que tienene un capital determinado en la actualidad o que prevén que van a ir acumulando a lo largo de los años, lo que que quiere es rentabilizar ese capital. En este caso, el mejor consejo sería realizar una \tb{gestión pasiva}, es decir, alguien que no tiene tiempo para dedicarse a seguir la bolsa diariamente o que tenga como fuente de ingresos otra actividad que no es la de inversor.
    \item \tb{Inversor a corto plazo},  son aquellas personas cuya principal fuente de ingresos la \ti{especulación bursátil} (también denominado especulador o trader), tiene que seguir una \ti{gestión activa}, estar pendiente de la evolución de las cotizaciones, de la revalorización de los diferentes activos financieros; tiene que saber cuando entrar y cuando salir en el mercado con qué cantidad, etc.
\end{itemize}

Nos centramos en las premisas del primer tipo de inversor, \ti{a largo plazo}:

\begin{enumerate}[label=\alph*)]
    \item Su objetivo es rentabilizar el capital. Cubrirse de la \ti{inflación} de forma que su capital al menos mantenga su poder adquisitivo. Introducimos dos conceptos que son:
    \begin{itemize}
        \item \tb{interés simple}
        \item \tb{interés compuesto}
    \end{itemize}
    \begin{testexample}[ Pongamos que queremos adquirir un vehículo, y que ese vehículo en la actualidad, tiene un coste de 10000€, ese sería su precio. Supongamos que la inflación crece a un ritmo de 2\% anual, es decir, el precio de los activos, incluido el de este vehículo, va a subir anualmente de manera acumulada un 2\%. ¿Cómo evolucionará, en 10 años. el coste del vehículo?]
        \begin{equation*}
            \begin{split}
                t &= 0 \rightarrow \euros{10.000} \\
                t &= 1 \rightarrow  (10.000 + 2\%) x 10.000 = 10.000 x (1 + 2\%) = \euros{10.200} \\
                t &= 2 \rightarrow (10.200 + 2\%) x 10.200 = 10.200 x (1+2\%) = 10.000 x (1+2\%)^2 = \euros{10.404} \\
                t &= 10 \rightarrow (10.000 x 1+2\%)^{10} = \euros{12.189,94}
            \end{split}
        \end{equation*}

        Hemos empleado la \ti{ley de interés compuesto} frente al \ti{interés simple}, puesto que lo que hacemos es que los incrementos se acumulan sobre los valores anteriores, es decir, el incremento en el precio del vehículo no es \ti{200€} al año, sino que es del 2\% siempre sobre el último valor considerado, que hemos calculado. De haber utilizado el \ti{interés simple} el resultado habría sido de \euros{10.400}.

        Podríamos hacernos la pregunta de ¿cuál será el poder adquisitivo de \euros{10.000} actuales dentro de 10 años?.
        \begin{equation*}
            \begin{split}
                t &= 0 \rightarrow \euros{10.000} \\
                t &= 1 \rightarrow \frac{10.000}{(1+2\%)} = \euros{9.803,92} \\
                t &= 3 \rightarrow \frac{9.803,92}{(1+2\%)} = \frac{10.000}{(1+2\%)^2} = \euros{9.611,68} \\
                t &= 10 \rightarrow \frac{10.000}{(1+2\%)^{10}} = \euros{8.203,48}
            \end{split}
        \end{equation*}
    \end{testexample}

    La inversión bursátil lo que nos permite es defendernos de la pérdida de poder adquisitivo dada por el efecto de la inflación.

    Si nuestro dinero no se invierte, perderemos poder adquisitivo por dos vías:
    \begin{enumerate}
        \item por el efecto de la inflación.
        \item por el coste de oportunidad, no haberlo rentabilizado a través de una inversión.
    \end{enumerate}

    Con el siguiente ejemplo podemos analizar el efecto de estas dos situaciones.

    \begin{testexample}[ Imaginemos disponer de \euros{50.000}]
        
    \end{testexample}
    
    \item Mejor opción, la \tb{renta variable}, invirtiendo que no se necesite, ni ahora ni en el corto/medio plazo.
    \item Diversificar.
    \item Reinvertir dividendos, siempre que se invierte en acciones, por el principio de que la bolsa en largo plazo es siempre alcista; reinvirtiendo los dividendos nos vamos a asegurar:
    \begin{itemize}
        \item una mayor rentabilidad
        \item una serie de ventajas fiscales.
    \end{itemize}
\end{enumerate}

\section{Renta variable en el largo plazo}

\section{Diversificación}


\section{El papel de los dividendos}